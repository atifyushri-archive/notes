\documentclass[a4paper,10pt]{article}
\usepackage[margin=0.5in]{geometry}
\usepackage[british]{babel}
\usepackage[T1]{fontenc}
\usepackage{amsmath}

\usepackage{multicol}
\makeatletter
\newenvironment{tablehere}{\def\@captype{table}}{}
\makeatother
\renewcommand{\arraystretch}{1.25}

\usepackage{titlesec}
\titleformat{\section}
	{\large\bfseries}
	{\thesection}{16pt}{}
\titleformat{\subsection}
	{\large\bfseries}
	{\thesubsection}{16pt}{}
\titleformat{\subsubsection}
	{\normalsize\bfseries}
	{\thesubsubsection}{16pt}{}

\usepackage{enumitem}
\setlist[itemize]{noitemsep, topsep=0pt}
\setlist[enumerate]{noitemsep, topsep=0pt}

\setlength{\parindent}{0pt}
\setlength{\multicolsep}{0pt}
\begin{document}
{\LARGE\bfseries\scshape Individuals, Firms, Markets and Market Failure}

\vspace*{1em}\hrule\vspace*{1em}

\begin{multicols}{2}
	\tableofcontents

	\section{Economic Methodology and the Economic Problem}
	\subsection{Economic Methodology}

	\subsubsection{Economics as a Social Science}
	Economics is a social science, meaning it concerns itself with the study of
	people's behaviour and society, the economy and its participants. Social
	sciences such as economics studies, interprets concepts and create models that
	measure the economy qualitatively rather than quantitatively proving a
	hypothesis.

	\subsubsection{Positive and Normative Statements}
	John Neville Keynes characterised \textbf{positive economics}, which deals with positive statements, as \textbf{"what is"} rather than \textbf{normative economics' "what it ought to be"}.
	\medskip

	To put it simply, \textbf{positive statements} refer to the \textbf{current state of the economy}, i.e. what the economy is doing. \textbf{Normative statement}, on the other hand, refers to what economists believe \textbf{what it should be}.
	\medskip

	For example,
	\begin{center}
		\parbox{0.75\linewidth}{"the UK economy is currently experiencing headline inflation of 10\% YoY"}
	\end{center}
	is a positive statements, it is \textbf{based on fact} and references the current state of the economy.
	\medskip

	However,
	\begin{center}
		\parbox{0.75\linewidth}{"headline inflation in the UK should be controlled to 2\% per year"}
	\end{center}
	is a normative statement as it \textbf{applies value judgements} onto an observation.

	\begin{center}
		\begin{tablehere}
			\begin{tabular}{|c|c|}
				\hline
				Positive Statements & Normative Statements \\
				\hline
			\end{tabular}
		\end{tablehere}

		\Huge TODO
	\end{center}

	Normative statements are often prescriptive to a situation while positive statements are descriptive of an event.
	\begin{align*}
		\text{\texttt{positive statement:}}  & \ \text{\texttt{event occurred, this is...}}
		\\
		\text{\texttt{normative statement:}} & \ \text{\texttt{event occurred, do this...}}
	\end{align*}

	\subsubsection{The Effect of Value Judgements}
	Value judgements are an assessment or evaluation, i.e. opinions, of certain behaviours or events within the economy. Economists often uses value judgements to make key decisions such as organisational policies. If a policy-maker aims for their policy to be effective, they employ value judgements of experts in order to successfully execute their intentions.
	\medskip

	Value judgements is a way for economists to make the correct decisions and predict the outcome. As positive consequences is often the objective, value judgements provides a way to extrapolate or predict possible consequences.

	\subsubsection{The 'Best' Option}
	There is no 'best' option. The best option varies by entity. The best option is often weighed out by certain criteria which includes: positive consequences of a policy, moral and political judgements. Some policies will have great positive consequences but comes at a cost of other factors such as moral alignment and political views. There is a myriad of 'best' options and it is important to categorise factors and evaluate which are the most useful.

	\subsection{The Nature and Purpose of Economic Activity}
	\subsubsection{The Central Purpose}
	The central purpose of economic activity is to satisfy needs and wants. The economy exists and functions for two parties to exchange their resources and meet their respective needs and wants. Producers needs and wants profit whilst consumers needs and wants goods.

	\subsubsection{The Key Economic Decisions}
	The economy runs on a set of important decisions. Speaking generally, an economy, which includes the producers, the consumers and the government, needs to decide on important economic decision, which are:
	\begin{enumerate}
		\item[---] what to produce?
		\item[---] how to produce?
		\item[---] who benefits from the production of the goods and services?
	\end{enumerate}
	In all of these matter, although things can be argued one way or another, each participant of the economy decides on an answer to each of these questions.
	\medskip

	Consumers demands a product, producers supplies said product, the government provisions the product from the producers to the consumers. Consumers meets their needs and wants, the producers make a profit and the government stays stable. Each economic agent have a say in these matter and each of them benefit in their own ways.

	\subsection{Economic Resources}
	\subsubsection{The Factors of Production}
	Economic resources are also called the factors of production as these facilitates the production of goods and services. There are four factors of production:
	\begin{enumerate}
		\item land
		\item labour
		\item capital
		\item enterprise
	\end{enumerate}
	It is worth noting that most economic resources come from the environment and it is a scare resource. Most resources that derives from the environment is non-replenishable, meaning there is only limited amount of it available.

	\subsection{Scarcity, Choice and the Allocation of Resources}
	\subsubsection{Fundamental Economic Problem}
	The economy exists for only one purpose: to satisfy needs and wants. The problem arises from the fact that resources are scarce and some even non-replenishable. This is a problem because needs and wants of economic participants are unlimited.

	\subsubsection{The Effect of Scarcity}
	Economic resources scarcity leads to choices that needs to be made on what to produce, how to produce and who will benefit from the production of said goods and services. Economic resources needs to be efficiently allocated to satisfy the most needs and wants.

	\subsubsection{The Problem of Opportunity Costs}
	Every choice made to allocate economic resources poses a problem: opportunity cost. Opportunity cost is the value of an alternative choice after it is forgone. This means that once a choice is made, there is a potential of positive consequences being sacrificed.

	\subsection{Production Possibility Diagrams}
\end{multicols}
\end{document}